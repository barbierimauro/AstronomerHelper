\documentclass[a4paper,12pt]{article}

\author{Mauro Barbieri}
\date{12 May 2023}
\title{ESO Archive: How to associate a RAW file with ADP and ANCILLARY files\\
      The case of HARPS}


\begin{document}
\maketitle


This workflow aims to establish a relationship between the RAW files (identified by 'dp\_id'), the ADP files (identified by 'archive\_id\_spec'), 
and the ancillary TAR files (identified by 'archive\_id\_anci') for the HARPS instrument.
Currently, no table provides this correspondence for any ESO instrument, making this operation essential for comprehensive data analysis. 
While the specific procedures outlined here focus on the HARPS instrument, the same technique could be applied to any other instrument,
making this workflow universally valuable for data from the ESO archive.

The entire operation sequence is designed to categorize, align, and extract pertinent data across multiple tables. 
The final output, 'finaltable', is a well-organized data table that comprises all observations, along with their associated metadata from all the involved tables. 
This table will be used subsequently for retrieving and using these files for keyword extractions (since they are FITS files, even inside the TAR) and data analysis on the ADP. 
This workflow hence serves as a precursor for more in-depth analysis and is especially beneficial when an entire dataset from an instrument is under scrutiny, 
  as opposed to files from individual observing programs that can be retrieved from the ESO archive.

Specific grouping based on 'dp\_tech' and 'dp\_type' is integral to distinguish between scientific data and calibration data. 
Sorting by date ('dateadp') and retaining only the most recent ADP file for each RAW file ensures that the dataset is current and consistent. 
Given that the 'phase3v2' table contains multiple ADP entries for the same RAW file, this process of sorting, identifying duplicates, and 
keeping only the first entry is critical to maintaining the dataset's integrity.

This workflow, though complex, is necessary and currently the only feasible method using TAP queries. 
It not only makes data handling more streamlined but also ensures the use of the most up-to-date and relevant files, thus optimizing the subsequent data analysis.

\begin{enumerate}
\item 
Start by querying the 'dbo.raw' table where the instrument name is 'HARPS'. This helps narrow down the entries to those relevant for the HARPS instrument.
\begin{verbatim}
select * from dbo.raw where instrument = 'HARPS'
\end{verbatim}
%select * from ivoa.obscore where instrument_name = 'HARPS'
%select * from phase3v2.files where name like '%harps%'
%select * from phase3v2.provenance where source_file like '%harps%'

\item 
    Make a subset science files from the 'dbo.raw' table based on the following conditions: if the 'dp\_tech' column starts with the string 'ECHELLE' and the 'dp\_type' column starts with 'STAR'.
\begin{verbatim}
startsWith(dp_tech,"ECHELLE") & startsWith(dp_type,"STAR")
\end{verbatim}
    
\item 
    Query the 'phase3v2.files' table for entries related to 'HARPS'.
\begin{verbatim}
select * from phase3v2.files where name like '%HARPS%'
\end{verbatim}

\item 
    In the result of the 'phase3v2.files' query, create two new columns: 'dpid1' and 'dateadp'. 'dpid1' is created by extracting a substring of the 'name' column. 
This new column will help in joining tables later on. 
    \begin{verbatim}
        dpid1 = startsWith(name,"H") ? 
                substring(name,0,29) : 
                substring(replaceAll(name,"TEST.",""),0,29)
    \end{verbatim}
    'dateadp' is created by converting a part of the 'archive\_id' column, which is in ISO format, to a Modified Julian Date (MJD) format. 
This new column will be useful for sorting operations based on the file creation date.
    \begin{verbatim}
        dateadp = isoToMjd(substring(archive_id,4,30))
    \end{verbatim}

\item 
    In the same result table, make two groups, 'spec' and 'anci', based on the 'category' value being either 'SCIENCE.SPECTRUM' or \\'ANCILLARY.HARPSTAR' respectively. 
These groups are useful for categorizing different types of data in the table.

\item 
    Next, sort the 'phase3v2.files' table by 'dateadp' in descending order. This organizes the data by the file creation date, with the most recent entries appearing first.

\item 
    With the sorted data, execute an internal match operation on 'dpid1' column for both the 'spec' and 'anci' groups separately. 
This operation matches rows with exact values in 'dpid1', grouping them together. 
From each group, only the first entry (which is also the most recent due to the previous sorting operation) is retained. 
The result of each operation is a new table, 'spec' and 'anci', respectively.

\item 
    Perform a crossmatch operation between 'dbo.raw' and the 'spec' table based on 'dp\_id' and 'dpid1' columns respectively. 
The crossmatch operation is akin to a SQL left outer join. For each row in 'dbo.raw', the best matching row from 'spec' is selected. 
The result is a new temporary table, 'temp1'.

\item 
    Similarly, perform a crossmatch operation between 'temp1' and the 'anci' table. 
The result of this operation is the table 'temp2', which contains all the observations with their associated files: the ADPs and the TARs.

\item 
    Query the 'ivoa.obscore' table for entries related to 'HARPS'.
\begin{verbatim}
select * from ivoa.obscore where instrument_name = 'HARPS'
\end{verbatim}

\item 
    Lastly, execute a crossmatch operation between 'temp2' and 'ivoa.obscore' tables. 
This operation is based on 'archive\_id\_spec' from the 'temp2' table and 'dp\_id' from 'ivoa.obscore'. 
The crossmatch operation is akin to a SQL left outer join, where for each row in 'temp2', the best matching row from 'ivoa.obscore' is selected. 
This operation results in the final table, 'finaltable'. The 'archive\_id\_spec' is the archive\_id from the 'spec' table that was matched with 'dbo.raw' in a previous operation.

\end{enumerate}

\end{document}
